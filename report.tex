\documentclass[12pt]{article}

% Packages
\usepackage[hidelinks]{hyperref}
\usepackage{fancyhdr}
\usepackage{titlesec}
\usepackage{url}

\newcommand{\HRule}{\rule{\linewidth}{0.5mm}}

% Header and Footer
\pagestyle{fancy}
\fancyhf{}
\lhead{[Name of the app/project] Report}
\rhead{Programming 2 (CM12005)}
\cfoot{\thepage}
\setlength{\headheight}{14.5pt}

% Adjust the spacing of the section titles
\titleformat{\section}{\Large\bfseries}{\thesection}{1em}{}
\titlespacing{\section}{0pt}{*0.5}{*0.5}

\begin{document}

% Title Page
\begin{titlepage}
    \centering
    {\Huge \bfseries [Name of the App/Project]}\\[0.2cm]
    \vspace*{0.5cm}
    {\large Author 1, Author 2, Author 3, Author 4, Author 5, Author 6, Author 7, Author 8, Author 9}\\[2cm] % Authors
    \vspace*{1cm}

    \begin{minipage}{0.9\textwidth}

        \textbf{Abstract:} About 200 words here. Lorem ipsum aliquam
        etiam erat velit scelerisque in dictum non consectetur a erat nam at lectus
        urna duis convallis convallis tellus id interdum velit laoreet id donec
        ultrices tincidunt arcu non sodales neque sodales ut etiam sit amet nisl
        purus in mollis nunc sed id semper risus in hendrerit gravida rutrum
        quisque non tellus orci ac auctor augue mauris augue neque gravida in
        fermentum et sollicitudin ac orci phasellus egestas tellus rutrum tellus
        pellentesque eu tincidunt tortor aliquam nulla facilisi cras fermentum odio
        eu feugiat pretium nibh ipsum consequat nisl vel pretium lectus quam id leo
        in vitae turpis massa sed elementum tempus egestas sed sed risus pretium
        quam vulputate dignissim suspendisse in est ante in nibh mauris cursus
        mattis molestie a iaculis at erat pellentesque adipiscing commodo elit at
        imperdiet dui accumsan sit amet nulla facilisi morbi tempus iaculis urna id
        volutpat lacus laoreet non curabitur gravida arcu ac tortor dignissim
        convallis aenean et tortor at risus viverra adipiscing at in tellus integer
        feugiat scelerisque varius morbi enim nunc faucibus a pellentesque sit amet
        porttitor eget dolor morbi non arcu risus quis varius quam quisque id diam
        vel quam elementum pulvinar etiam non quam lacus suspendisse faucibus
        interdum posuere lorem.

    \end{minipage}
\end{titlepage}
\thispagestyle{fancy}

\newpage

\tableofcontents
\thispagestyle{empty}

\newpage

\setcounter{page}{1}

\section{Introduction (2 pages)}

\textbf{An account of the problem your software system is intended to address, including a
description of the problem area (Personal Informatics) supported by relevant literature (See
NFR3.1 and NFR3.2), what can make a PI software system effective for its users, and a clear
statement of the main idea behind your particular PI software system.}

Increasingly, medical advice and research is showing that minor changes to an individual’s daily routines can have positive effects on their overall health, wellbeing and productivity. For example we are often reminded of the advantages of walking 10,000 steps (Tudor-Locke et al., 2011), sleeping for at least 7 hours (Watson et al., 2015) and drinking at least 6 cups of water (NHS, 2023) every day. However, as discussed by Madore and Wagner (2019), processing multiple tasks concurrently can lead to reduced efficiency. Therefore, tools to aid in the tracking and pursuing of these personal development goals are likely to help an individual reach their goals more effectively.\par

PI(Personal Informatics) software is software designed to help collect and analyse data about an individual with the aim of promoting self-understanding and betterment. A PI system could be an ideal solution to this issue as they could be very useful in allowing an individual to monitor and streamline their progress towards their goals for self-improvement.\par

However, in designing a PI system it is important to ensure user engagement; if a user stops using a system then the system can’t help them. Kersten-van Dijk et al. (2017) found that the studies they reviewed which discussed users dropping out of using a PI system reported dropout rates of 7-44\%. This suggests that a significant proportion of people might not feel motivated to continue using PI software. Furthermore, Jones and Kelly (2018) found that presenting users with too much information could leave them feeling overwhelmed; Rapp and Cena (2016) found that first-time users of PI systems could find the act of recording their data burdensome; and Potapov et al. (2021) discovered that teenagers often found PI systems either controlling or confusing depending on the way they were designed.\par

Fortunately, some of these studies, as well as many others, have investigated how to make a PI software system effective for its users and proposed ways of designing systems with this in mind. For instance, Jones and Kelly (2018) suggested that, to avoid overwhelming users, a system should only show the user information that is interesting to them. They found that interesting information was that which was surprising, useful or statistically significant. It was also found that users found it more interesting when insights could be provided between aspects of their life, rather than within them. Rapp and Cena (2016) recommended providing users with tailored summaries and targets; control over their data; and reviews of past data. In this way they suggested that feelings of freedom and nostalgia might be fostered, increasing positive views and connections with the software. Potapov et al. emphasise the importance of finding an appropriate balance between constraints to orient the user towards positive goals and freedom to allow them to use the tool in a way that suits them. Additionally, a study by Loerakker et al. (2023) showed that promoting self-compassion through the design of a Personal Informatics system can foster positive self-reflection while reducing the chances of rumination and cessation of goal pursuit. The study showed that framing data positively can promote self-compassion. For example, highlighting positive achievements rather than criticising poorer performance.\par

Our intention is to create a PI system aimed at current university students.  The metrics we intend to track are hours of sleep, steps taken, hours spent productively and water intake. We arrived at this list of metrics by means of a survey and interviews which gauged interest in different metrics within our target audience. Our survey revealed that 60\% of respondents were most interested in tracking their health , followed by 30\% who were most interested in tracking their own learning and development. Our system will allow manual data entry. Some metrics will also use automatic data collection via API to make using the software easier. Additionally, it will allow users to set personal goals and see useful data about their performance. This data will be able to showcase any correlation in each metric against time individually and between separate metrics. This functionality will all take place within a desktop application with a graphical user interface. We hope that, by tracking these four important aspects of student life, our system will provide interesting and useful insights that guide our users towards better health and productivity while ensuring that they aren't overwhelmed by too much data or the need to balance their goals.


\section{Agile Software Process Planning and Management (2 pages)}

This section must describe how you carried out your project as an Agile process, including
how you planned and tracked your sprints, and how Scrum meetings were used to manage
the evolution of your idea and the features you decided to include in your software system.


\section{Specification of Software Requirements (5 pages)}

This section should include:

An account of how you established your detailed system requirements (building on
the initial requirements).

A description of the specific domain of your personal informatics system, linked to
the viewpoint of students as your target users.

Use cases to illustrate the services your PI system is intended to offer to relevant
actors, including target users.

A set of functional and non-functional requirements for your system, organised into
sensible groups and following the guidance given you in lectures (e.g. relationships
and priorities).


\section{Design (5 pages)}

This section should include:

A set of UML models describing the low-level design of your system (class models,
sequence diagrams, etc.)

A justification of your chosen design. This should explicitly show how your design
meets both functional and non-functional requirements, making direct reference
them and the rationale behind your design as it embodies your PI idea.

Make sure you include explanatory text with your design models: the meaning of a diagram
Page 9 of 11
is very rarely self-evident because it depends on an explanatory frame for interpretation. So
you must make sure to express in words what you wish the reader to understand by them.


\section{Software Testing (Verification) (2 pages)}

This section should include:

Testing plans, indicating how you planned to perform verification of your PI system.
These plans should reflect your requirements and design work, and be completed
before any implementation (coding) work begins.

Evidence of testing - test case results.


\section{Reflection and Conclusion (4 pages)}

This section should include two main sections:

A critique of your software system’s requirements specification, design and testing,
including what might be improved and why you feel it would be better with these
changes.

A critical reflection on the group’s software process, including evidence of Agility in
your software process, evidence of having evolved your requirements to reflect
changes in understanding of problem and viability of designs.


\section{References}

\renewcommand{\refname}{} 
\vspace{-20pt}
\begin{thebibliography}{99}
    % EXAMPLE 1
    \bibitem{ref-ex1} Surname, X. \textit{Article name} [Online].
    Available from:~\url{https://google.com}.

    % EXAMPLE 2
    \bibitem{ref-ex2} AfterAcademy, 2020. \textit{Sudoku Solver} [Online].
    Available from:~\url{https://afteracademy.com/blog/sudoku-solver/}. See the
    \textit{Complexity Analysis} section.
\end{thebibliography}


\section{Appendices}
You must include your one-page final GCF as the first appendix to your report. You should
include records of your group meetings (minutes) as the second appendix. You may include
transcripts of interviews and/or other evidence of primary research you have conducted as
the third appendix.
Also, you may include earlier design models to evidence how your ideas have evolved as an
appendix. if you wish to include, as well as any unit tests or additional test runs you might
wish to include with your report

\end{document}
