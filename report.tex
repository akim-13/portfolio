\documentclass[12pt]{article}

% Packages
\usepackage[hidelinks]{hyperref}
\usepackage{fancyhdr}
\usepackage{titlesec}
\usepackage{url}

\newcommand{\HRule}{\rule{\linewidth}{0.5mm}}

% Header and Footer
\pagestyle{fancy}
\fancyhf{}
\lhead{[Name of the app/project] Report}
\rhead{Programming 2 (CM12005)}
\cfoot{\thepage}
\setlength{\headheight}{14.5pt}

% Adjust the spacing of the section titles
\titleformat{\section}{\Large\bfseries}{\thesection}{1em}{}
\titlespacing{\section}{0pt}{*0.5}{*0.5}

\begin{document}

% Title Page
\begin{titlepage}
    \centering
    {\Huge \bfseries [Name of the App/Project]}\\[0.2cm]
    \vspace*{0.5cm}
    {\large Author 1, Author 2, Author 3, Author 4, Author 5, Author 6, Author 7, Author 8, Author 9}\\[2cm] % Authors
    \vspace*{1cm}

    \begin{minipage}{0.9\textwidth}

        \textbf{Abstract:} About 200 words here. Lorem ipsum aliquam
        etiam erat velit scelerisque in dictum non consectetur a erat nam at lectus
        urna duis convallis convallis tellus id interdum velit laoreet id donec
        ultrices tincidunt arcu non sodales neque sodales ut etiam sit amet nisl
        purus in mollis nunc sed id semper risus in hendrerit gravida rutrum
        quisque non tellus orci ac auctor augue mauris augue neque gravida in
        fermentum et sollicitudin ac orci phasellus egestas tellus rutrum tellus
        pellentesque eu tincidunt tortor aliquam nulla facilisi cras fermentum odio
        eu feugiat pretium nibh ipsum consequat nisl vel pretium lectus quam id leo
        in vitae turpis massa sed elementum tempus egestas sed sed risus pretium
        quam vulputate dignissim suspendisse in est ante in nibh mauris cursus
        mattis molestie a iaculis at erat pellentesque adipiscing commodo elit at
        imperdiet dui accumsan sit amet nulla facilisi morbi tempus iaculis urna id
        volutpat lacus laoreet non curabitur gravida arcu ac tortor dignissim
        convallis aenean et tortor at risus viverra adipiscing at in tellus integer
        feugiat scelerisque varius morbi enim nunc faucibus a pellentesque sit amet
        porttitor eget dolor morbi non arcu risus quis varius quam quisque id diam
        vel quam elementum pulvinar etiam non quam lacus suspendisse faucibus
        interdum posuere lorem.

    \end{minipage}
\end{titlepage}
\thispagestyle{fancy}

\newpage

\tableofcontents
\thispagestyle{empty}

\newpage

\setcounter{page}{1}

\section{Introduction (2 pages)}
Increasingly, medical advice and research is showing that minor changes to an individual’s daily routines can have positive effects on their overall health, wellbeing and productivity. For example, we are often reminded of the advantages of walking 10,000 steps (Tudor-Locke et al., 2011), sleeping for at least 7 hours (Watson et al., 2015) and drinking at least 6 cups of water (NHS, 2023) every day. However, as discussed by Madore and Wagner (2019), processing multiple tasks concurrently can lead to reduced efficiency. Therefore, tools to aid in the tracking and pursuing of these personal development goals are likely to help an individual reach their goals more effectively.\par

PI(Personal Informatics) software is software designed to help collect and analyse data about an individual with the aim of promoting self-understanding and betterment. A PI system could be an ideal solution to this issue as it could be very useful in allowing an individual to monitor and streamline their progress towards their goals for self-improvement.\par

However, in designing a PI system it is important to ensure user engagement; if a user stops using a system then the system cannot help them. Kersten-van Dijk et al. (2017) found that the studies they reviewed which discussed users dropping out of using a PI system reported dropout rates of 7-44\%. This suggests that a significant proportion of people might not feel motivated to continue using PI software. Furthermore, Jones and Kelly (2018) found that presenting users with too much information could leave them feeling overwhelmed; Rapp and Cena (2016) found that first-time users of PI systems could find the act of recording their data burdensome; and Potapov et al. (2021) discovered that teenagers often found PI systems either controlling or confusing depending on the way they were designed.\par

Fortunately, some of these studies, as well as many others, have investigated how to make a PI software system effective for its users and proposed ways of designing systems with this in mind. For instance, Jones and Kelly (2018) suggested that, to avoid overwhelming users, a system should only show the user information that is interesting to them. They found that 'interesting information' was that which was surprising, useful or statistically significant. It was also found that users found it more interesting when insights could be provided between aspects of their life, rather than within them. Rapp and Cena (2016) recommended providing users with tailored summaries and targets; control over their data; and reviews of past data. In this way they suggested that feelings of freedom and nostalgia might be fostered, increasing positive views and connections with the software. Potapov et al. emphasise the importance of finding an appropriate balance between constraints to orient the user towards positive goals and freedom to allow them to use the tool in a way that suits them. Additionally, a study by Loerakker et al. (2023) showed that promoting self-compassion through the design of a Personal Informatics system can foster positive self-reflection while reducing the chances of rumination and cessation of goal pursuit. The study showed that framing data positively can promote self-compassion. For example, highlighting positive achievements rather than criticising poorer performance.\par

Our intention is to create a PI system aimed at current university students. To  decide on a list of metrics for our system to track, we investigated the interests of our chosen demographic by means of a Google Forms survey. The survey, which was sent to students online, revealed that 60\% of respondents were most interested in tracking their health, followed by 30\% who were most interested in tracking their own learning and development. In addition, when asked to rate their interest in tracking their data for self-improvement on a scale of 1-10, 55\% of participants rated their interest at 8 or above. This reinforced our belief that a PI system would be popular among our target audience. We also conducted some individual interviews of students involving more specific questions about health and desirable PI system features. These interviews revealed an interest in graphs and a simple user interface which we will take into account when designing our application. All the students we interviewed also expressed a desire to alter the number of hours of sleep they got on a daily basis. Based on these investigations, as well as research into Fitbit and Garmin software, we decided that our system would track hours of sleep, water intake, steps taken and hours spent productively.\par

Our system will allow data to be entered manually. Some metrics will also be able to collect data automatically via the Fitbit API to reduce the burden of data entry. Additionally, our software will allow users to set personal goals and see useful data about their performance. This data will be able to showcase correlation in each metric against time individually and between separate metrics. This functionality will all be accessed via a desktop application with a graphical user interface. We hope that, by tracking these four important aspects of student life, our system will provide interesting and useful insights that guide our users towards better health and productivity while ensuring that they are not overwhelmed by too much data or the need to balance their goals.


\section{Agile Software Process Planning and Management (2 pages)}
\textbf{This section must describe how you carried out your project as an Agile process, including
how you planned and tracked your sprints, and how Scrum meetings were used to manage
the evolution of your idea and the features you decided to include in your software system.}\par
To carry out this project in an efficient and organised manner, we followed Agile practices and values using the Scrum framework. This involved using a series of week-long sprints to break up and work through the continuously evolving product backlog. Between each sprint, a meeting was held to discuss the outcome of the previous sprint, go over any changes to the product backlog, produce a sprint backlog and divide this sprint backlog between the group members for completion during the next sprint. During sprints, group members worked either individually or as part of a smaller group to complete their allocated element of the sprint backlog. Regular scrum meetings were also used to allow group members to communicate progress, plans, obstacles and key information. We assigned a scrum master to gain a full understanding of Agile values and methodologies and ensure the group's adherence to them. A product owner was also chosen to be in charge of managing the product backlog and organising sprints.\par

Our first sprint was devoted towards research into PI systems and what students wanted in one. Three group members were assigned to researching PI systems, two to market research, two to gauging interest in students, one to compiling the initial project backlog and one to researching the most appropriate programming language to use for the project. By the end of the week, all members had fulfilled the duties assigned to them. The research was discussed as a group and used to produce the idea for our system. As this idea evolved, it was converted into a list of functional requirements and added to the product backlog.\par

The second sprint consisted of designing and beginning to create our application. A member of the team set up a framework for our project on GitHub and two team members were assigned the task of creating a UI (User Interface) diagram for our application. Having done this, the UI designers presented the result in a scrum meeting and, based on this, other members of the group created a coding plan, started creating the UI and implemented some basic functionality. Meanwhile, a group member worked on creating a database and demonstrating the use of it; another researched and showcased the use of a suitable API; and the remaining people began the writing of the report. The scrum meetings held throughout the week were key in ensuring that all the programmers could stay abreast of any important changes made by others with impacts on their own work, as well as allowing the sharing and discussion of issues. At the end of the sprint, all available team members attended a Google Meet meeting in which all progress was discussed. The sprint was very successful, with every individual achieving their designated goals. Based on the API research conducted, it was decided that the Fitbit API was most suitable for our system. This evolution of the idea of our project meant that the product backlog was updated.\par

The sprint that followed largely prioritised the implementation of desired functionalities from the product backlog. Five members of the group were individually allocated to specific aspects of the program. This included implementing the productive time tracker, achievements, goals, graphs and the Fitbit API integration. The other available group members were given sections of the report to work on. During the frequent scrum meetings in this sprint, reporting back was important to maintain a full understanding of the latest changes among the programmers.


\newpage
\section{Specification of Software Requirements}

This section breaks down the core requirements for our Personal Informatics
(PI) system, focusing on helping students keep up with their fitness and study
goals. We utilised feedback from surveys, talks with users, market research,
and research articles to figure out what features our system should have. Our
aim is to make a system that assists students in improving both their physical
health and studying habits whilst not being overly invasive.

\subsection{Establishing Detailed System Requirements}

We started by asking people what they are looking for in an personal
improvement app. We found out that most students use PI systems for
tracking their fitness (60\%) and their learning progress (30\%). This
helped us get a good idea of what our app should focus on. Next, we looked
at what users said about our initial ideas as well as comparing our ideas
to expert findings from our listed articles. People told us they like apps
that are intuitive but still allow for great insight and control into goals
and targets. They also wanted to connect with friends through the app and
liked the idea of seeing all their information in a single area.\par

After this, we made a final list of what our app needs to do. We will
ensure our system is good at tracking health and study time, allows users
to share their progress with friends, and keeps all their data safe and
private.
 
\subsection{Specific Domain of Application}

Our PI system is intricately designed with a student demographic at its core,
addressing their unique challenges such as managing academic deadlines
alongside maintaining a balanced lifestyle.\par 

Students grapple with the dual challenges of academic accountability and
physical well-being. The system shall provide a suite of tools for effective
study habits and simple health tracking.
 
\subsection{Use Cases}

Our app is mainly for people who need assistance managing their time healthily.
We are aware that many people have difficulty maintaining a healthy work-life
balace. Our app aims to help them track key aspects of the student lifestyle.
The ways this will be done are as follows: 

\begin{itemize}

    \item \textbf{Tracking sleep time}: The system will track the time a user
        sleeps each night. This will be used to establish an average sleep time
        and patterns in their sleep schedule which will then be used to suggest
        ways to improve this. For example: if a user's average sleeping time is
        below 7 hours, more sleep could be recommended.
    
    \item \textbf{Tracking step count}: The system will receive inputs of a user's daily
        step count either from the fitbit API or manual input. This can then be
        used to analyse whether a user is moving regularly or performing enough
        exercise. This goal should be adjusted to ensure a user is not
        consistently overachieving or underachieving their goal.

    \item \textbf{Tracking work done}: The system will receive manual inputs of how much
        time a user has spent studying. This will then be compared to past data
        to suggest whether the user has studied enough or should do more
        studying that day. The data collected over time should be used to set
        study goals for a given period of time.
    
    \item \textbf{Tracking water intake}: The system will receive inputs of how much
        water a user has drank in a day.

    \item \textbf{Challenges}: Users can partake in challenges either individually or
        with friends. They can choose the difficulty of the challenges which
        can be changed daily if a user finds it too challenging or not engaging
        enough. These could be user set or generated based on habits. 

\end{itemize}

 The majority of data is going to be pulled from the fitbit API upon request for each user. We must ensure that potential users without the relevant hardware can still use our system. This should be done in an intuitive way. This will also allow for easier gathering of data fields such as water intake which can be hard to track through physical equipment like a fitbit. Additionally, not all fields we have selected can be accessed directly via the fitbit API and as such manual entry is necessary.\newline

To ensure that our users understand what data we are storing, each user shall be allowed to view the data we have stored on them for each of these key fields.\newline

\subsection{Allow Users to Manage Goals}
This will be done through "achievements" generated by the system or manually set by the user. They can either be individual or group goals. Group goals will consist of a friendly competiton to see who can perform the best in a given field. For example this could be a friendly challenge between flatmates. The system should have suggestions to users in setting their goals and the time frame to make it suitable to the users, or else they might lose interest over time if they set all the goals too difficult and could not finish any of it.\\ 

Additionally, a user should have the option to share their achievements with other users, although this may be better achieved through means not involving our application.
***idk if we want to do this so feel free to remove it***\\ 
    
\subsection{Comparing User Data}
Our system should ensure that a user can compare all data fields in an intuitive manner. This will be done via graphs that allow the user to select what fields they wish to view along with a time frame (day, week, month to date).\\

From a user's perspective, seeing a clear change in habits can be both a great motivator to improve as well as allowing them to identify variables unknown to our system that may be affecting their data. From our system's perspective, this data is necessary to allow us to suggest methods of which to improve their lifestyle.\\ 

By including time frames, the user can identify potential outside factor that our system cannot consider. For example, a negative event in a given time frame such as a coursework deadline could influence sleep count. They could then use this data to come up with ways to improve their behaviour patterns for other similar circumstances.\\ 


***will be expanding this, should fit to 5 pages***

Non-Functional Requirements: \\ 

This section encompasses many other factors that our system must consider:

-    Scalability: It should be capable of handling an increase in users and additional features over time.\\

-    Security: User data must be kept secure, with privacy settings that users can control.\\

-    Performance: The app needs to be responsive and reliable, providing a smooth user experience.\\


*** I'd say we don't need to mention this side of the requirements in our report, instead evidencing/discussing it in other sections***\\

-    Follow SCRUM framework: We have sprints and meeting every week to make sure everything go smoothly, and we have a Scrum Master to make sure that the progress is in the right way.\\

-    Expanding initial requirements: pending\\

-    Background research: We did articles research, however, most of them are in the PI field, which narrows down our view on how to build the system. It is helpful if we do more research outside of this field which might give us more idea about functional requirements and how to implement them.\\

-       Testing: pending\\

*** let me know if you want me to fully write this section out though***



\section{Design (5 pages)}

This section should include:

A set of UML models describing the low-level design of your system (class models,
sequence diagrams, etc.)

A justification of your chosen design. This should explicitly show how your design
meets both functional and non-functional requirements, making direct reference
them and the rationale behind your design as it embodies your PI idea.

Make sure you include explanatory text with your design models: the meaning of a diagram
Page 9 of 11
is very rarely self-evident because it depends on an explanatory frame for interpretation. So
you must make sure to express in words what you wish the reader to understand by them.


The class diagram, derived from our database Entity-Relationship Diagram (ERD), illustrates the intricate structure of our Personal Informatics system, designed specifically for tracking and managing various user-centric activities. This diagram provides a visual representation of the data relationships and is critical in guiding the development process to ensure the system meets both functional and non-functional requirements effectively.\par

User Class: Central to our system, the User class captures essential personal identifiers and account information for each user. Attributes include Username, Email, and Password, which are fundamental for ensuring secure user authentication and system access. This class acts as the primary entity with which other classes associate, establishing a one-to-many relationship across various data-tracking entities. Each instance of a User can associate with multiple records in subordinate classes, reflecting the diverse activities and metrics tracked by the system.\par

UserAchievements Class: This class stores specific achievements of users, linked directly to the User class. Attributes like Username, Date, ActivityID, TimeFrameID, and Value allow for detailed recording and analysis of user accomplishments across different activities and timeframes. It enables the system to provide feedback and insights based on historical achievement data, supporting motivational features such as goal setting and progress tracking.\par

UserStep Class: Dedicated to tracking the physical activity of steps taken, this class includes attributes such as Username, Date, and Steps. By recording daily step counts, the UserStep class feeds into the system’s health monitoring and fitness tracking capabilities, allowing users to set and monitor physical activity goals.\par

UserSleep Class: Focusing on sleep habits, this class records the duration of sleep per day for each user with attributes Username, Date, and SleepHour. It is vital for analyzing sleep patterns and correlating them with other health metrics, which is essential for offering personalized health insights and improving user well-being.\par

UserWork Class: This class tracks work and study sessions, an essential feature for our student users. Attributes include Username, WorkName, Date, and Endtime, facilitating the tracking of academic and work-related activities over time. This functionality supports students in managing their time and productivity more effectively.\par

UserWater Class: Monitoring hydration, the UserWater class includes attributes such as Username, Date, and LitresDrank. Hydration tracking is crucial for overall health, and this class allows the system to remind users to stay hydrated and track their daily water intake.\par

UserGoals Class: As a strategic component of our system, the UserGoals class holds data related to the personal objectives of users. With attributes like Username, ActivityID, TimeFrameID, Date, and Value, this class supports the setting and monitoring of personalized goals, enhancing the system's capability to drive user engagement and encourage behavior modification.\par



\section{Software Testing (Verification) (2 pages)}

This section should include:

Testing plans, indicating how you planned to perform verification of your PI system.
These plans should reflect your requirements and design work, and be completed
before any implementation (coding) work begins.

Evidence of testing - test case results.


\section{Reflection and Conclusion (4 pages)}

This section should include two main sections:

A critique of your software system’s requirements specification, design and testing,
including what might be improved and why you feel it would be better with these
changes.

A critical reflection on the group’s software process, including evidence of Agility in
your software process, evidence of having evolved your requirements to reflect
changes in understanding of problem and viability of designs.


\section{References}

\renewcommand{\refname}{} 
\vspace{-20pt}
\begin{thebibliography}{99}
    % EXAMPLE 1
    \bibitem{ref-ex1} Surname, X. \textit{Article name} [Online].
    Available from:~\url{https://google.com}.

    % EXAMPLE 2
    \bibitem{ref-ex2} AfterAcademy, 2020. \textit{Sudoku Solver} [Online].
    Available from:~\url{https://afteracademy.com/blog/sudoku-solver/}. See the
    \textit{Complexity Analysis} section.

    \bibitem{jones-2018}Jones, S. L. and Kelly, R., 2018. 
    Dealing With Information Overload in Multifaceted Personal Informatics Systems. 
    \textit{Human–computer interaction}, 33(1), pp. 1–48. \url{https://doi.org/10.1080/07370024.2017.1302334}

    \bibitem{kersten-van-Dijk-2017}Kersten-van Dijk, E. T., Westerink, J. H. D. M., Beute, F. and 
    IJsselsteijn, W. A., 2017. 
    Personal Informatics, Self-Insight, and Behavior Change: A Critical Review of Current Literature. 
    \textit{Human–computer interaction}, 32(5–6), pp. 268–296.

    \bibitem{potapov-2021}Potapov, K., Vasalou, A., Lee, V. and Marshall, P., 2021. 
    What do Teens Make of Personal Informatics? 
    Young People's Responses to Self-Tracking Practices for Self-Determined Motives. 
    \textit{Proceedings of the 2021 CHI Conference on Human Factors in Computing Systems}, 
    8-13 May 2021, New York. Association for Computing Machinery, pp. 1–10.

    \bibitem{loerakker-2023}Loerakker, M.B., Niess, J., Bentvelzen, M. and Woźniak, P.W., 2023. 
    Designing Data Visualisations for Self-Compassion in Personal Informatics. 
    \textit{Proceedings of the ACM on interactive, mobile, wearable and ubiquitous technologies}, 7(4), pp. 1–22.

    \bibitem{madore-2019}Madore, K. P. and Wagner, A. D., 2019. Multicosts of Multitasking. 
    \textit{Cerebrum:the dana forum on brain science}[Online], 4(19). Available from:
    \url{https://www.ncbi.nlm.nih.gov/pmc/articles/PMC7075496/#sec-a.c.ftitle} 
    [Accessed 5 April 2024]

    \bibitem{nhs-2023}National Health Service, 2023. \textit{Water, drinks and hydration} [Online] 
    Available from: 
    \url{https://www.nhs.uk/live-well/eat-well/food-guidelines-and-food-labels/water-drinks-nutrition} 
    [Accessed 5 April 2024]
    
    \bibitem{rapp-2016}Rapp, A. and Cena, F., 2016. Personal informatics for everyday life: 
    How users without prior self-tracking experience engage with personal data. 
    \textit{International journal of human-computer studies}, 94, pp. 1-17

    \bibitem{tudor-locke-2011}Tudor-Locke, C., Craig, C. L., Brown, W. J., Clemes, S. A., De Cocker, K., 
    Giles-Corti, B., Hatano, Y., Inoue, S., Matsudo, S. M., Mutrie, N., Oppert, J. M., Rowe, D. A., 
    Schmidt, M. D., Schofield, G. M., Spence, J. C., Teixeira, P. J., Tully, M. A. and Blair, S. N., 2011. 
    How many steps/day are enough? For adults. 
    \textit{The international journal of behavioral nutrition and physical activity} 
    [Online], 8(79). Available from: \url{https://doi.org/10.1186/1479-5868-8-79}

    \bibitem{watson-2015}Watson, N. F., Badr, M. S., Belenky, G., Bliwise, D. L., Buxton, O. M., Buysse, D., 
    Dinges, D. F., Gangwisch, J., Grandner, M. A., Kushida, C., Malhotra, R. K., Martin, J. L., Patel, S. R., 
    Quan, S. F. and Tasali, E., 2015. 
    Recommended Amount of Sleep for a Healthy Adult: 
    A Joint Consensus Statement of the American Academy of Sleep Medicine and Sleep Research Society. 
    \textit{Sleep}, 38(6), pp.843–844.
\end{thebibliography}


\section{Appendices}
You must include your one-page final GCF as the first appendix to your report. You should
include records of your group meetings (minutes) as the second appendix. You may include
transcripts of interviews and/or other evidence of primary research you have conducted as
the third appendix.
Also, you may include earlier design models to evidence how your ideas have evolved as an
appendix. if you wish to include, as well as any unit tests or additional test runs you might
wish to include with your report

\end{document}
