\documentclass[12pt]{article}

% Packages
\usepackage[hidelinks]{hyperref}
\usepackage{fancyhdr}
\usepackage{titlesec}
\usepackage{url}

\newcommand{\HRule}{\rule{\linewidth}{0.5mm}}

% Header and Footer
\pagestyle{fancy}
\fancyhf{}
\lhead{[Name of the app/project] Report}
\rhead{Programming 2 (CM12005)}
\cfoot{\thepage}
\setlength{\headheight}{14.5pt}

% Adjust the spacing of the section titles
\titleformat{\section}{\Large\bfseries}{\thesection}{1em}{}
\titlespacing{\section}{0pt}{*0.5}{*0.5}

\begin{document}

% Title Page
\begin{titlepage}
    \centering
    {\Huge \bfseries [Name of the App/Project]}\\[0.2cm]
    \vspace*{0.5cm}
    {\large Author 1, Author 2, Author 3, Author 4, Author 5, Author 6, Author 7, Author 8, Author 9}\\[2cm] % Authors
    \vspace*{1cm}

    \begin{minipage}{0.9\textwidth}

        \textbf{Abstract:} About 200 words here. Lorem ipsum aliquam
        etiam erat velit scelerisque in dictum non consectetur a erat nam at lectus
        urna duis convallis convallis tellus id interdum velit laoreet id donec
        ultrices tincidunt arcu non sodales neque sodales ut etiam sit amet nisl
        purus in mollis nunc sed id semper risus in hendrerit gravida rutrum
        quisque non tellus orci ac auctor augue mauris augue neque gravida in
        fermentum et sollicitudin ac orci phasellus egestas tellus rutrum tellus
        pellentesque eu tincidunt tortor aliquam nulla facilisi cras fermentum odio
        eu feugiat pretium nibh ipsum consequat nisl vel pretium lectus quam id leo
        in vitae turpis massa sed elementum tempus egestas sed sed risus pretium
        quam vulputate dignissim suspendisse in est ante in nibh mauris cursus
        mattis molestie a iaculis at erat pellentesque adipiscing commodo elit at
        imperdiet dui accumsan sit amet nulla facilisi morbi tempus iaculis urna id
        volutpat lacus laoreet non curabitur gravida arcu ac tortor dignissim
        convallis aenean et tortor at risus viverra adipiscing at in tellus integer
        feugiat scelerisque varius morbi enim nunc faucibus a pellentesque sit amet
        porttitor eget dolor morbi non arcu risus quis varius quam quisque id diam
        vel quam elementum pulvinar etiam non quam lacus suspendisse faucibus
        interdum posuere lorem.

    \end{minipage}
\end{titlepage}
\thispagestyle{fancy}

\newpage

\tableofcontents
\thispagestyle{empty}

\newpage

\setcounter{page}{1}

\section{Introduction (2 pages)}

An account of the problem your software system is intended to address, including a
description of the problem area (Personal Informatics) supported by relevant literature (See
NFR3.1 and NFR3.2), what can make a PI software system effective for its users, and a clear
statement of the main idea behind your particular PI software system.


\section{Agile Software Process Planning and Management (2 pages)}

This section must describe how you carried out your project as an Agile process, including
how you planned and tracked your sprints, and how Scrum meetings were used to manage
the evolution of your idea and the features you decided to include in your software system.


\section{Specification of Software Requirements (5 pages)}

This section should include:

An account of how you established your detailed system requirements (building on
the initial requirements).

A description of the specific domain of your personal informatics system, linked to
the viewpoint of students as your target users.

Use cases to illustrate the services your PI system is intended to offer to relevant
actors, including target users.

A set of functional and non-functional requirements for your system, organised into
sensible groups and following the guidance given you in lectures (e.g. relationships
and priorities).


\section{Design (5 pages)}

This section should include:

A set of UML models describing the low-level design of your system (class models,
sequence diagrams, etc.)

A justification of your chosen design. This should explicitly show how your design
meets both functional and non-functional requirements, making direct reference
them and the rationale behind your design as it embodies your PI idea.

Make sure you include explanatory text with your design models: the meaning of a diagram
Page 9 of 11
is very rarely self-evident because it depends on an explanatory frame for interpretation. So
you must make sure to express in words what you wish the reader to understand by them.


\section{Software Testing (Verification) (2 pages)}

This section should include:

Testing plans, indicating how you planned to perform verification of your PI system.
These plans should reflect your requirements and design work, and be completed
before any implementation (coding) work begins.

Evidence of testing - test case results.


\section{Reflection and Conclusion (4 pages)}

This section should include two main sections:

A critique of your software system’s requirements specification, design and testing,
including what might be improved and why you feel it would be better with these
changes.

A critical reflection on the group’s software process, including evidence of Agility in
your software process, evidence of having evolved your requirements to reflect
changes in understanding of problem and viability of designs.


\section{References}

\renewcommand{\refname}{} 
\vspace{-20pt}
\begin{thebibliography}{99}
    % EXAMPLE 1
    \bibitem{ref-ex1} Surname, X. \textit{Article name} [Online].
    Available from:~\url{https://google.com}.

    % EXAMPLE 2
    \bibitem{ref-ex2} AfterAcademy, 2020. \textit{Sudoku Solver} [Online].
    Available from:~\url{https://afteracademy.com/blog/sudoku-solver/}. See the
    \textit{Complexity Analysis} section.
\end{thebibliography}


\section{Appendices}
You must include your one-page final GCF as the first appendix to your report. You should
include records of your group meetings (minutes) as the second appendix. You may include
transcripts of interviews and/or other evidence of primary research you have conducted as
the third appendix.
Also, you may include earlier design models to evidence how your ideas have evolved as an
appendix. if you wish to include, as well as any unit tests or additional test runs you might
wish to include with your report

\end{document}
