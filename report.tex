\documentclass[12pt]{article}

% Packages
\usepackage[hidelinks]{hyperref}
\usepackage{geometry}
\usepackage{fancyhdr}
\usepackage{graphicx}
\usepackage{titlesec}
\usepackage{url}

% Page Geometry
\geometry{a4paper, margin=0.7in, headheight=14.5pt}

% Header and Footer
\pagestyle{fancy}
\fancyhf{}
\lhead{Sudoku Solver Code Report}
\rhead{CM12001}
\cfoot{\thepage}

% Custom Title Page
\newcommand{\HRule}{\rule{\linewidth}{0.5mm}}

% Adjust the spacing of the section titles
\titleformat{\section}{\Large\bfseries}{\thesection}{1em}{}
\titlespacing{\section}{0pt}{*0.5}{*0.5} 

\begin{document}

\begin{titlepage}
    \centering
    \vspace*{\stretch{1}}
    \HRule \\[0.4cm]
    \vspace{0.5cm}
    {\Huge \bfseries Sudoku Solver}\\[0.5cm]
    {\Large Code Report}\\[0.4cm]
    \HRule \\[1.5cm]
    
    % \textsc{\Large Akim Komarnitskii}\\[0.5cm]
    \textsc{\Large Artificial Intelligence}\\[0.5cm]
    \textsc{CM12001}\\[2cm]
    
    {\large \today}\\[3cm]

    \vspace*{\stretch{2}}
\end{titlepage}

\thispagestyle{fancy}

\section{Difficulty of Approach}
\label{difficulty}

The Sudoku solver developed for this project presents a balanced approach,
embodying both moderate complexity and methodical ambition. It incorporates two
core techniques: constraint propagation and recursive depth-first search
coupled with time-based monitoring. The sophistication of the algorithm is
further enhanced by strategic optimizations. These include employing two Sudoku
solving strategies, utilizing two search optimization heuristics, and
leveraging efficient data structures for effective game representation and
operation. Although each technique, when considered independently, is
relatively straightforward, the overall complexity of the algorithm arises from
numerous interactions between the mechanisms, many of which are of recursive
nature. The intricacies of the described techniques and optimizations are
elaborated in Sections \ref{description} and \ref{complexity}.

\section{Description of Algorithm}
\label{description}

The game states are represented by the \texttt{grid} dictionary, with cell
coordinates as keys, and strings of available digits as values \cite{norvig}.
The algorithm is based on two primary concepts: constraint propagation and
recursive depth-first search (RDFS). The former is leveraged to iteratively and
recursively eliminate available digits for each cell based on Sudoku rules. For
instance, when a cell is determined to have only one valid digit, this digit is
then recursively removed from the list of possible values for all cells in the
cell's \texttt{peers}. This action can trigger further changes in the
\texttt{peers} of these \texttt{peers}, effectively narrowing down the search
space. However, to ensure algorithmic completeness, constraint propagation is
complemented by RDFS, which selects the optimal cell and digit for elimination
using heuristics. Each decision initiates a new search branch. If a branch
leads to a dead end, the \texttt{ContradictionException} is raised, the branch is
immediately discarded, and another path is explored. This process continues
until a solution is found or all possibilities are exhausted, ensuring a
complete and efficient solution.

\section{Optimizations and Complexity}
\label{complexity}

While the space complexity remains constant due to the fixed size of the grid
and the data structures employed to represent it, the theoretical time
complexity is calculated as \( O(9^n) \), where \( n \) is the number of empty
spaces, and 9 is the branching factor (the number of possible digits)
\cite{bigO}. However, in practice, this exponential complexity is significantly
mitigated by the implementation of strategic heuristics and optimizations. The
minimum remaining values heuristic selects the cell with the fewest available
digits, thereby reducing the average branching factor. The least constraining
values heuristic further refines the digit selection by choosing a digit that
affects the least number of the cell's \texttt{peers}. Furthermore, the
\textit{hidden singles} strategy assigns digits to cells where only one valid
option exists within a \texttt{unit}, and the \textit{naked twins} strategy
eliminates specific digits from peers within a \texttt{unit}, based on pairs of
cells holding identical digit pairs. Lastly, a time constraint of 29.99 seconds
is applied, after which \texttt{INVALID\_SOLUTION} is returned. This approach
is based on empirical observations suggesting that puzzles requiring more than
this duration are typically either too complex to resolve within a reasonable
time frame or lack a viable solution.

\section{Reflections and Suggestions for Further Work}
\label{future}

Although the algorithm demonstrates notable efficiency, it occasionally
encounters puzzles that challenge the predefined time threshold (see Section
\ref{complexity}), suggesting there is still room for improvement in terms of
strategies, optimizations and implementation methods. Future enhancements might
focus on integrating advanced heuristics or applying machine learning
algorithms to refine the search strategy. Exploring parallel processing could
offer a significant speedup by concurrently exploring multiple solution
branches. Additionally, adjusting the time threshold dynamically based on
puzzle complexity could result in more nuanced management of edge cases.
Extensive testing across various puzzle difficulties could further enhance the
algorithm's robustness and overcome its limitations.

\begin{thebibliography}{9}

    \bibitem{norvig} Norvig, P. \textit{Solving Every Sudoku Puzzle} [Online].
    Available from:~\url{https://norvig.com/sudoku.html}. See the rationale for
    using strings in \texttt{grid} and data structures for \texttt{units} and
    \texttt{peers} under the sections \textit{Sudoku Notation and Preliminary
    Notions} and \textit{Analysis}.

    \bibitem{bigO} AfterAcademy, 2020. \textit{Sudoku Solver} [Online].
    Available from:~\url{https://afteracademy.com/blog/sudoku-solver/}. See the
    \textit{Complexity Analysis} section.

\end{thebibliography}

\end{document}
